\chapter{Podsumowanie}
\section{Wnioski}
Na zbadanych plikach można zauważyć, że najlepiej sprawdził się algorytm Betweenness. Zaimplementowany wcześniej Personalized PageRank sprawdził się zdecydowanie najsłabiej. Pozostałe algorytmy sprawowały się w miarę porównywalnie jeśli chodzi o skuteczność ujednoznaczniania. Wspomniana skuteczność to ilość poprawnie ujednoznacznionych wyrazów do wszystkich wyrazów do ujednoznacznienia. Należy wspomieć, że długość wykonania najlepszego algorytmu jest jednak bardzo długa (2,5 raza dłuższa od PageRank), co skutecznie ograniczyło ilość wybranych plików do testów. Mimo zastosowania ograniczenia grafu na którym został uruchomiony Betweenness czas ten był długi. Ograniczenie to polega na wycięciu części grafu na podstawie wyniku PageRank, co gwarantuje dłuższy czas wykonania (muszą zostać uruchomione 2 algorytmy). Z racji braku dostępu do szybszego sprzętu nie można było uzyskać lepszego pokrycia dostarczonych, anotowanych przykładów dokumentów (przykłady obejmowały ok 1600 dokumentów z których wybrano tylko 5).
