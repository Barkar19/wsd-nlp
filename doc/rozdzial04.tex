\chapter{Komponenty projektu}
\section{Skrypty pomocnicze}
Na potrzeby projektowe zrealizowano kilka  skryptów pomocniczych, które realizują akwizycję danych testowych. Poniżej przedstawiono krótki opis każdego z nich.
\begin{itemize}
\item youtube.sh - pobieranie zadanego filmu z Internetu
\item split-into-frames.sh - dzielenie pliku multimedialnego na osobne klatki
\item movie-cut.sh - wycinanie fragmentu filmu z całości
\item tile.sh - układanie obrazów w mozaikę (na potrzeby niniejszego dokumentu)
\end{itemize}
\section{Program}
Program wykonujący algorytm opisany w \ref{algo} jest programem konsolowym, który przyjmuje następujące argumenty:
\begin{itemize}
\item help,h - wyświetlenie pomocy
\item input,i - lokalizacja pliku wejściowego lub folderu z sekwencją zdjęć
\item config,c - lokalizacja pliku konfiguracji algorytmu
\item output,o - nazwa pliku wyjściowego
\item verbose,v - wyświetlanie informacji związanych z przetwarzaniem
\end{itemize}
\newpage
\section{Plik konfiguracyjny XML}
Program działa na podstawie pliku konfiguracyjnego w postaci xml, który zawiera informacje o procesie przetwarzania. Poniżej przedstawiono opis znaczników w  tymże pliku.
\begin{itemize}
\item <config> - znacznik główny
\item <preprocessing> - znacznik operacji wykonywanych przed rozpoczęciem głównego algorytmu. W jego ramach znajdują się:
\begin{itemize}
\item <resize> - dokonanie zmianu rozmiaru obrazu (atrybut 'value' o wartości rzeczywistej - skali).
\item <match> - wykonanie dopasowania zdjęć względem siebie (atrybut 'value' o wartościach true/false)
\item <colorspace> - wybór przestrzeni kolorów (atrybut 'value' o wartościach bgr/hls)
\end{itemize}
\item <processing> - znacznik zawierający opis przetwarzania serii pikseli. W jego ramach znajdują się wyłącznie znaczniki <function>
\item <function> zawierające kolejno funkcje w potoku. Zawiera następujące atrybuty:
\begin{itemize}
\item atrybut 'size' - wielkość rozpatrywanego podzbioru pikseli na którym wykonywana będzie funkcja
\item atrybut 'parameter' - opcjonalny parametr dla danej funkcji
\end{itemize}
\end{itemize}