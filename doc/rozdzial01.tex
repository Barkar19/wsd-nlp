\chapter{Wstęp}
\section{Opis problemu}
Ujednoznacznianie znaczeń słów (ang. word sense disambiguation) jest jednym z kluczowych problemów analizy semantycznej. Polega na wyznaczeniu znaczenia danego słowa w kontekście, tak aby cały kontekst był spójny. Znaczenia słów opisane są przez tak zwane synsety (zbiory synonimów, ang. synonyms sets). W tym celu stosuje się algorytmy nadzorowane i nienadzorowane. W algorytmach opartych o uczenie nadzorowane stosowane są metody uczenia maszynowego np. sieci neuronowe lub wektory maszyn wspierających. W uczeniu nienadzorowanym stosowane są zewnętrzne źródła wiedzy np. słowniki, słowosieci, grafy synsetów itp. 
\section{Idea metody}
W niniejszej pracy zaprezentowano rozwiązanie oparte o algorytm nienadzorowanym. Wykorzystano słowosieć oraz graf synsetów. W tym celu zaimplementowano rozszerzenia do platformy WoSeDon, która realizuje problem ujednoznaczniania sensów słów za pośrednictwem algorytmów opartych o PageRank. Rozszerzenia skupiają się na wykorzystaniu innych miar centralności w grafach m.in. betweenness, closeness, eigenvector. Poniżej przedstawiono szczegółowy opis metody.