\chapter{Platforma testowa}
\section{Komponenty}
Do realizacji projektu, wykorzystano szerg gotowych rozwiązań, które znacznie przyspieszyły proces implementacji. Poniżej przedstawiono istotne biblioteki, komponenty oraz elementy użyte w projekcie:
\begin{itemize}
\item WoSeDon - program bazowy
\item Wrocław CFR Tagger - tager 
\item corpus2 - biblioteka dostarczająca struktur danych i metod do obsługi anotowanych korpusów.
\item Korpus PWr - anotowany korpus służący do testów
\item docker - środowisko uruchmieniowe WoSeDona
\end{itemize}

Głównym elementem system jest WoSeDon, program służący do ujdnoznaczniania znaczeń słów za pomocą algorytmu PageRank. Niniejszy projekt sprowadził się do zaimplementowania rozszerzeń do WoSeDona wykorzystujących inne metody pomiaru centralności.
\section{Docker}
Ze względu na dosyć żmudny proces tworzenia środowiska testowego, zadecydowano o użyciu docekra, który pozwoli na szybkie tworzenie homogenicznego środowiska na wielu niezależnych od siebie maszynach. W ramach projektu powstały dwa typy środowisk opisanych przez poniższe dockerfile'e:
\begin{itemize}
\item Dockerfile-arch
\item Dockerfile-ubuntu
\end{itemize}

Utworzenie dwóch odrębnych dockerfile'i było wymuszone, ze względu na długi proces budowy środowiska opartego o Linkusa Arch (brak skompilowanych komponentów, długotrwałe kompilacje). Aby ominąć problem czasowy podjęto decyzję o utworzeniu środowiska opartego o Linuksa Ubuntu.